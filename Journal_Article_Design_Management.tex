%-----------------Examples of Journals to Submit Articles from Classroom Lectures-----------------------%
%-------------------------------------------------------------------------------------------------------%


\section{Journals}

\begin{enumerate}
\item Applied Mathematical Letters \cite{key1}
\item Computation Statistics and Data Analysis \cite{key2}
\item Statistics and Probability Letters \cite{key3}
\item Neural Networks \cite{key4}
\item Discrete Applied Mathematics \cite{key5}
\item Economic Modeling \cite{key6}
\end{enumerate}


\section{Journal Article Management}

\inlinecode{R}{Table.1<-c(ID,Journal,Title,Date.Submitted, Date.Reviewed, Revision.Number, Date.Accepted,Comment)}

\begin{table}[ht]
\caption{Journal Article Management for Classroom}
\begin{tabular}{rlllllll}
\hline
ID & Journal & Title & Date.Submitted & Date.Reviewed & Revision.Number & Date.Accepted & Comment \\
\hline
1 & \textbf{Applied Mathematical Letters} \cite{key1} & \textbf{Title 1} & "2000-01-01" & "2000-01-01" & 1 & "2000-01-01" & \\
\hline
\end{tabular}
\end{table}


\section{Tables}


\section{Figures}



\section{References}

\subsection{Cancer Journals}

\subsection{Biology Journals}

\subsection{Chemistry Journals}

\subsection{Botany Journals}

\subsection{Math Journals}

\bibitem[1]{key1} Applied Mathematical Letters
\newblock https://www.journals.elsevier.com/applied-mathematics-letters

\bibitem[1]{key2}Computational Statistics and Data Analysis
\newblock https://www.journals.elsevier.com/computational-statistics-and-data-analysis

\bibitem[1]{key3}Statistics and Probability Letters
\newblock https://www.journals.elsevier.com/statistics-and-probability-letters

\bibitem[1]{key4}Neural Networks
\newblock https://www.journals.elsevier.com/neural-networks

\bibitem[1]{key5}Discrete Applied Mathematics
\newblock https://www.journals.elsevier.com/discrete-applied-mathematics

\bibitem[1]{key6}Economic Modeling
\newblock https://www.journals.elsevier.com/economic-modelling



