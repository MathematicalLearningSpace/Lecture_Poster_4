%-----------------Example Tables for Students to Modify in the Classroom------------------------------------------------%
%-----------------These examples can be used in the pharmaeconometric sections of the lectures and journal articles-----%
%-----------------------------------------------------------------------------------------------------------------------%

%-----------------------------Example of a semantic numeric vector of keywords------------------------------------------
\begin{enumerate}
	\item INNOVATION, IMITATION and Market Structure in Pharmaceuticals \cite{key400}
	\item General Micro Evolutionary and Agents with Bounded Rationality History-friendly models (HFMs) EXAMINE HIGHER Complexity than OTHER Models \cite{key400}
	\item Strategies drug DISCOVERY from random GENERATED libraries to rational DESIGN \cite{key500}
	\item Movement from large MAXIMUM DIVERSE libraries to SMALL “drug-like” libraries. \cite{key500}
\end{enumerate}

%-----------------Example of Table Design----------------------- 
\section{Pharmaceutical Immuno-Oncology Portfolio Object}

\begin{table}
	\caption{Mono-Clonal Anti-Bodies for the Immuno-Oncology Portfolio \cite{key8001} }
	\begin{tabular}{r|p{4cm}|l}
		Compound & Company & Comment \\
		\hline
		Revlimid & Celgene & \\
		Opdivo & Bristol-Myers Squibb Ono Pharmaceutical & \\
		Imbruvica & AbbVie (Pharmacyclics) Johnson and Johnson & \\
		Keytruda & Merck and Company & \\
		Ibrance & Pfizer & \\
		Tecentriq & Roche & \\
		Darzalex & Johnson and Johnson & \\
		Perjeta & Roche & \\
		Xtandi & Astellas Pharma and Pfizer & \\
		Avastin & Roche & \\
		Herceptin & Roche & \\
		Gazyva & Roche & \\
		Jakafi & Novartis & \\
		Venclexta & Roche & \\
		Rituxan & Roche, Pharmstandard & \\
		\hline
	\end{tabular}
\end{table}

\section{Materials and Methods}
%-----------------------------Example of a semantic numeric vector of keywords------------------------------------------
\begin{enumerate}
	\item METHOD INVENT systems NOT components of systems. \cite{key1000}
	\item SELECT PROBLEM domain in KNOWLEDGE Space.\cite{key1000}
	\item INVENTED ECONOMIC system of lighting. \cite{key1000}
	\item INVENT by REPETITION with ADAPTATION to complex environments by INCREMENTAL SMALL CHANGE to APPROXIMATE final USE.\cite{key1000}
	\item BLENDED Economics with INVENT. \cite{key1000}
	\item ASSEMBLED and ORGANIZED the RESOURCES. \cite{key1000}
	\item EFFECTIVE DRAWING. \cite{key1000}
	\item ABILITY ADAPT to COMPLEX change. \cite{key1000}
	\item METAPHOR to INVENT. \cite{key1000}
	\item INVENTION, ENGINEERING and MANUFACTURING are Different. \cite{key1000}
\end{enumerate}
%-----------------------------Example of a semantic numeric vector of keywords------------------------------------------
\begin{enumerate}
	\item Value-at-Risk (VaR) is CLASSICAL DOWNSIDE risk MEASURE.\cite{key100}
	\item RETURNS are Data TRANSFORMATIONS. \cite{key100}
	\item IF RETURNS with SKEWNESS AND OR EXCESS KURTOSIS, then Cornish-Fisher MEASURES. \cite{key100}
	\item VaR DECOMPOSITIONS total portfolio VaR into risk PERCENTAGE of EACH portfolio COMPONENT. \cite{key100}
	\item Value-at-Risk (VaR) for (a) univariate, (b) component, (c) marginal cases by DIFFERENT METHODS. \cite{key100}
	\item Expected Shortfall (ES) OR Conditional Value at Risk (CVaR) OR Expected Tail Loss (ETL) for (a) univariate, (b) component (c) marginal cases by DIFFERENT METHODS. \cite{key100}
	\item DEFAULT probability level INTERVAL 1 < a < 5 percent, ES of RETURNS is negative VALUE of expected VALUE of RETURN when RETURN less a-quantile.  \cite{key100}
	\item Conditional value-at-risk (CVaR) is risk MEASURE COHERENT AND CONVEX function of portfolio WEIGHTS. \cite{key100}
	\item MAXIMIZE portfolio RETURN OR MINIMIZE portfolio SECOND MOMENT OR MAXIMIZE QUADRATIC utility OR MINIMIZE ETL subject to (a) leverage, (b) box, (c) group or category, (d) position limit, (e) target FIRST MOMENT return AND/OR (f) factor exposure constraints on WEIGHTS (g) RISK Aversion Parameter.\cite{key100}
\end{enumerate}
%--------------------------Visualization Charts and Diagrams------------------------------------------
\begin{table}
	\caption{Visualization Of Portfolio Metrics \cite{key100},\cite{key104}}
	\begin{tabular}{r|p{4cm}|ll}
		\hline
		ViewKeyID & Name & Description & Comment \\
		\hline
		1 & ACF plus & ACF OR ACF by PACF 2 PANEL & \\
		2 & Bar & BAR chart of RETURNS  & \\
		3 & Bar VaR & Periodic RETURNS BAR chart by risk METRIC & \\
		4 & Box Plot & Box Whiskers & \\
		5 & Capture Ratios & Capture Ratios by BENCHMARK  & \\
		6 & Correlation & Correlation matrix & \\
		7 & Cumulative Returns & Cumulates Periodic RETURNS  & \\
		8 & Draw down &	Time Series of Drawdowns  & \\
		9 & ECDF & ECDF by Normal CDF  & \\
		10 & Events & Time Series of ALIGNED EVENTS  & \\
		11 & Histogram	&  Histogram of RETURNS  & \\
		12 & QQ Plot & QQ chart  & \\
		\hline
		13 & Regression & SCATTER of RETURNS by Market BENCHMARK & \\
		14 & Relative Performance &	Relative PERFORMANCE of N RETURNS  & \\
		15 & Risk Return Scatter & Scatter RETURNS by RISK for N Instrument COMPARISON & \\
		\hline
		16 & Rolling Correlation &	Rolling Correlation of N RETURNS  & \\
		17 & Rolling Mean &	Rolling First Moment RETURN  & \\
		18 & Rolling Performance &	Rolling Performance METRICS & \\
		19 & Rolling Quantile Regression &	Relative Temporal Regression by Performance & \\
		20 & Rolling Regression & Relative Regression Performance by TIME  & \\
		\hline
		21 & Scatter &	Scatter by Defaults  & \\
		22 & Snail Trail &	Risk by RETURN by Rolling TIME INTERVALS  & \\
		23 & Stacked Bar &	Stacked BAR  & \\
		24 & VaR Sensitivity &	Sensitivity of Value-at-Risk or Expected Shortfall   & \\
		\hline
		25 & AR	 & Characteristic ROOTS: ARIMA model  & \\
		26 & Arima & Characteristic ROOTS: ARIMA model  & \\
		27 & Bats & Components : BATS model  & \\
		28 & ETS & Components : ETS model  & \\
		29 & Forecast & Forecast plot  & \\
		30 & M Forecast & Multivariate Forecast  & \\
		31 & Spline Forecast & 	Forecast plot  & \\
		\hline
	\end{tabular}
\end{table}


\begin{table}[ht]
	\caption{Symbols Movement Indicator \cite{key100}}
	\begin{tabular}{rlll}
		\hline
		ID & Indicator & Stock Symbols A & Comment \\
		\hline
		101 & Monthly Downside Risk & 0 & \\ 
		102 & Annualised Downside Risk & 0 & \\ 
		103 & Downside Potential & 0 & \\ 
		104 & Upside Potential & 0 & \\ 
		105 & Upside Potential ratio & 0 & \\ 
		200 & Omega & 0 &  \\ 
		300 & Sortino ratio & 0 & \\ 
		400 & Omega-sharpe ratio & 0 & \\ 
		\hline
	\end{tabular}
\end{table}
#--------------------------Examples of Ratio Designs-------------------------------
\begin{table}[ht]
	\caption{Portfolio Movement Indicator for Downside Risk Summary \cite{key400}}
	\begin{tabular}{r|p{4cm}|ll}
		\hline
		ID & Indicator & Portfolio A & Comment \\
		\hline
		500 & Semi Deviation & 0.00 & \\ 
		501 & Gain Deviation & 0.00 & \\ 
		502 & Loss Deviation & 0.00 &  \\ 
		\hline
		601 & Downside Deviation (MAR=10\%) & 0.00 & \\ 
		602 & Downside Deviation (Rf=0\%) & 0.00 & \\ 
		603 & Downside Deviation (0\%) & 0.00 & \\ 
		\hline
		700 & Maximum Drawdown & 0.00 & \\ 
		\hline
		401 & Historical VaR (95\%) & 0.00 & \\ 
		402 & Historical ES (95\%) &  0.00 &  \\ 
		\hline
		411 & Modified VaR (95\%) &  0.00 & \\ 
		412 & Modified ES (95\%) &  0.00 &  \\ 
		\hline
	\end{tabular}
\end{table}


\section{References}

\subsection{Methodology}

\bibitem[1]{key1000}Wikipedia contributors. (2017, September 12). 
\newblock Edisonian approach. 
\newblock In Wikipedia, The Free Encyclopedia. Retrieved August 3, 2018, from  https://en.wikipedia.org/w/index.php?title=Edisonian_approach&oldid=800292477

\subsection{Cheminformatics}

\bibitem[1]{key7000}Luca Scrucca (2013). 
\newblock GA: A Package for Genetic Algorithms in R. 
\newblock Journal of Statistical Software, 53(4), 1-37. URL http://www.jstatsoft.org/v53/i04/.

\bibitem[1]{key7001}Luca Scrucca (2016). 
\newblock On some extensions to GA package: hybrid optimisation, parallelisation and islands evolution. Submitted to R Journal. 
\newblock Pre-print available at arXiv URL http://arxiv.org/abs/1605.01931.

\bibitem[1]{key7002}Hans Werner Borchers (2017). 
\newblock pracma: Practical Numerical Math Functions. 
\newblock R package version 2.0.7. https://CRAN.R-project.org/package=pracma

\bibitem[1]{key7003}Cao Y, Charisi A, Cheng L, Jiang T and Girke T (2008). 
\newblock “ChemmineR: a compound mining framework for R.”
\newblock Bioinformatics, *24*(15), pp. 1733-1734. doi: 10.1093/bioinformatics/btn307 (URL:http://doi.org/10.1093/bioinformatics/btn307), <URL:https://doi.org/10.1093/bioinformatics/btn307>.

\bibitem[1]{key7004}Kevin Horan and Thomas Girke (2017). 
\newblock ChemmineOB: R interface to a subset of OpenBabel functionalities. 
\newblock R package version 1.14.0. https://github.com/girke-lab/ChemmineOB

\bibitem[1]{key7005} Wang Y, Backman TWH, Horan K and Girke T (2013). 
\newblock “fmcsR: mismatch tolerant maximum common substructure searching in R.” 
\newblock Bioinformatics, *29*(21), pp. 2792-2794. doi:10.1093/bioinformatics/btt475 (URL: http://doi.org/10.1093/bioinformatics/btt475), <URL:https://doi.org/10.1093/bioinformatics/btt475>.

\subsection{Combinatorial Chemistry}

\bibitem[1]{key500}Richard Pommier Swanson (2004)
\newblock The Entrance of Informatics into Combinatorial Chemistry
\newblock American Society for Information Science and Technology


\subsection{Immuno-Oncology}

\bibitem[2]{key8001}Special Report (2017)
\newblock Top 15 Best Selling Cancer Drugs 2022
\newblock Retrieved July 31, 2018, from https://www.fiercepharma.com/special-report/special-report-top-15-best-selling-cancer-drugs-2022

\subsection{Econometrics}

\bibitem[3]{key400} Christian Garavaglia, Franco Malerba, Luigi Orsenigo, Michele Pezzoni (2014)
\newblock Innovation and Market Structure in Pharmaceuticals An Econometric Analysis on Simulated Data
\newblock Jahrbücher für Nationalökonomie und Statistik 

\subsection{R API}

\bibitem[1]{key100} Brian G. Peterson and Peter Carl (2018). 
\newblock PerformanceAnalytics:Econometric Tools for Performance and Risk Analysis. 
\newblock R package version 1.5.2.https://CRAN.R-project.org/package=PerformanceAnalytics

\bibitem[1]{key101}Csardi G, Nepusz T: 
\newblock The igraph software package for complex network research,
\newblock InterJournal, Complex Systems 1695. 2006. http://igraph.org

\bibitem[1]{key102}Douglas Bates and Martin Maechler (2017). 
\newblock Matrix: Sparse and Dense Matrix Classes and Methods. 
\newblock R package version 1.2-11. https://CRAN.R-project.org/package=Matrix

\bibitem[1]{key103}David B. Dahl (2016). 
\newblock xtable: Export Tables to LaTeX or HTML. 
\newblock R package version 1.8-2. https://CRAN.R-project.org/package=xtable

\bibitem[1]{key104}Hyndman RJ (2017). 
\newblock forecast: Forecasting functions for time series and linear models. 
\newblock R package version 8.1, <URL:http://github.com/robjhyndman/forecast>.

\bibitem[1]{key105}Hyndman RJ and Khandakar Y (2008).
\newblock Automatic time series forecasting: the forecast package for R.” 
\newblock Journal of Statistical Software_, *26*(3), pp. 1-22.

\bibitem[1]{key106}David Shaub and Peter Ellis (2017).
\newblock forecastHybrid: Convenient Functions for Ensemble Time Series Forecasts. 
\newblock R package version 1.1.9. https://CRAN.R-project.org/package=forecastHybrid

\bibitem[1]{key107}Raviv E (2015). 
\newblock ForecastCombinations:Forecast combinations in R. 
\newblock R package version 1.1, <URL:https://CRAN.R-project.org/package=ForecastCombinations>.

