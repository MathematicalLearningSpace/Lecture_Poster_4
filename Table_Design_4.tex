%--------Example Resuable Table Design Pattern for Mathematical Notes by Students in the Classroom--------------------------
\section{Metabolities}

\centering
\begin{table}[H]\footnotesize
	\caption{Metabolites}
	\begin{tabular}{rp{1cm}p{4cm}p{1cm}p{1cm}}
		\hline
		ID & Metabolite & Description & Comment & Reference \\
		\hline
		1 & PGA & 3-phosphoglyceric acid & & \cite{key1},\cite{key2} \\
		2 & BPGA & 2,3-bisphosphoglyceric acid & & \cite{key1},\cite{key2} \\
		3 & GAP & glyceraldehyde  3-phosphate & & \cite{key1},\cite{key2} \\
		4. & FBP & fructose 1,6-bisphosphate; & & \cite{key1},\cite{key2} \\
		5. & F6P & fructose 6-phosphate  & & \cite{key1},\cite{key2} \\
		6. & G6P & glucose 6-phosphate  & & \cite{key1},\cite{key2} \\
		7. & G1P & glucose I -phosphate & & \cite{key1},\cite{key2} \\
		8. & SBP & sedoheptulose 1,7-bisphosphate  & & \cite{key1},\cite{key2} \\
		9. & S7P & sedoheptulose 7-phosphate & & \cite{key1},\cite{key2} \\
		10. & E4P & erythrose 4-phosphate & & \cite{key1},\cite{key2} \\
		11. & X5P & xylulose 5-phosphate & & \cite{key1},\cite{key2} \\
		12. & R5P & ribose 5-phosphate & & \cite{key1},\cite{key2} \\
		13. & Ru5P & ribulose 5-phosphate & & \cite{key1},\cite{key2} \\
		14. & RuBP & ribulose 1,5-bisphosphate & & \cite{key1},\cite{key2} \\
		\hline
	\end{tabular}
\end{table}
\raggedright

\section{Calvin-Benson Cycle Enzymes}

\centering
\begin{table}[H]\footnotesize
	\caption{Calvin -Benson Cycle Enzymes}
	\begin{tabular}{rp{1cm}p{2cm}p{3cm}p{1cm}}
		\hline
		ID & Enzymes & Description & Reaction & Reference \\
		\hline
		1 & EC 3.1.3.11 & Fructose bisphosphatase & $FBP + H2O->F6P + Pi$ &\cite{key4000} \\
		2 & EC 3.1.3.37 &  sedoheptulose bisphosphatase &$SBP + H2O->S7P + Pi$ &\cite{key4000} \\
		3 & EC 2.7.1.19 & ribulose-5-phosphate kinase & $Ru5P + ATP->RuBP + ADP$ & \cite{key4000} \\
		4 & EC 4.1.1.39 &  ribulosebisphosphate carboxylase & $RuBP + CO2+ H2O->2 PGA$ & \cite{key4000} \\
		5 & EC 2.7.7.27 &  ADP-glucose pyrophosphorylase  &$G1P + ATP + H2O->ADPG + 2 Pi$ & \cite{key4000} \\
		\hline
	\end{tabular}
\end{table}
\raggedright


\section{References}


\subsection{Metabolities}

\bibitem[1]{key1} Bassham, J. A.  and Krause, G. H. (1969) 
\newblock Biochim. Biophys. Acta 19. 

\bibitem[1]{key2}Bassham, James A. (1972)
\newblock CONTROL OF PHOTOSYNTHETIC CARBON METABOLISM
\newblock Lawrence Berkeley National Laboratory https://escholarship.org/uc/item/1xz4n2h5


\subsection{KEGG}

\bibitem[400]{key4000} Kanehisa, Furumichi, M., Tanabe, M., Sato, Y., and Morishima, K.; 
\newblock KEGG: new perspectives on genomes, pathways, diseases and drugs. 
\newblock Nucleic Acids Res. 45, D353-D361 (2017).

\bibitem[401]{key4001} Kanehisa, M., Sato, Y., Kawashima, M., Furumichi, M., and Tanabe, M.; 
\newblock KEGG as a reference resource for gene and protein annotation. 
\newblock Nucleic Acids Res. 44, D457-D462 (2016).

\bibitem[402]{key4002} Kanehisa, M. and Goto, S.; 
\newblock KEGG: Kyoto Encyclopedia of Genes and Genomes. 
\newblock Nucleic Acids Res. 28, 27-30 (2000). 
